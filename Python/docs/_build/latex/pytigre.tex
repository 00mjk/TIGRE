%% Generated by Sphinx.
\def\sphinxdocclass{report}
\documentclass[letterpaper,10pt,english]{sphinxmanual}
\ifdefined\pdfpxdimen
   \let\sphinxpxdimen\pdfpxdimen\else\newdimen\sphinxpxdimen
\fi \sphinxpxdimen=.75bp\relax

\PassOptionsToPackage{warn}{textcomp}
\usepackage[utf8]{inputenc}
\ifdefined\DeclareUnicodeCharacter
% support both utf8 and utf8x syntaxes
\edef\sphinxdqmaybe{\ifdefined\DeclareUnicodeCharacterAsOptional\string"\fi}
  \DeclareUnicodeCharacter{\sphinxdqmaybe00A0}{\nobreakspace}
  \DeclareUnicodeCharacter{\sphinxdqmaybe2500}{\sphinxunichar{2500}}
  \DeclareUnicodeCharacter{\sphinxdqmaybe2502}{\sphinxunichar{2502}}
  \DeclareUnicodeCharacter{\sphinxdqmaybe2514}{\sphinxunichar{2514}}
  \DeclareUnicodeCharacter{\sphinxdqmaybe251C}{\sphinxunichar{251C}}
  \DeclareUnicodeCharacter{\sphinxdqmaybe2572}{\textbackslash}
\fi
\usepackage{cmap}
\usepackage[T1]{fontenc}
\usepackage{amsmath,amssymb,amstext}
\usepackage{babel}
\usepackage{times}
\usepackage[Bjarne]{fncychap}
\usepackage{sphinx}

\fvset{fontsize=\small}
\usepackage{geometry}

% Include hyperref last.
\usepackage{hyperref}
% Fix anchor placement for figures with captions.
\usepackage{hypcap}% it must be loaded after hyperref.
% Set up styles of URL: it should be placed after hyperref.
\urlstyle{same}
\addto\captionsenglish{\renewcommand{\contentsname}{Contents:}}

\addto\captionsenglish{\renewcommand{\figurename}{Fig.\@ }}
\makeatletter
\def\fnum@figure{\figurename\thefigure{}}
\makeatother
\addto\captionsenglish{\renewcommand{\tablename}{Table }}
\makeatletter
\def\fnum@table{\tablename\thetable{}}
\makeatother
\addto\captionsenglish{\renewcommand{\literalblockname}{Listing}}

\addto\captionsenglish{\renewcommand{\literalblockcontinuedname}{continued from previous page}}
\addto\captionsenglish{\renewcommand{\literalblockcontinuesname}{continues on next page}}
\addto\captionsenglish{\renewcommand{\sphinxnonalphabeticalgroupname}{Non-alphabetical}}
\addto\captionsenglish{\renewcommand{\sphinxsymbolsname}{Symbols}}
\addto\captionsenglish{\renewcommand{\sphinxnumbersname}{Numbers}}

\addto\extrasenglish{\def\pageautorefname{page}}

\setcounter{tocdepth}{1}



\title{pytigre Documentation}
\date{Apr 30, 2019}
\release{}
\author{Reuben Lindroos, Ander Biguri}
\newcommand{\sphinxlogo}{\vbox{}}
\renewcommand{\releasename}{}
\makeindex
\begin{document}

\pagestyle{empty}
\sphinxmaketitle
\pagestyle{plain}
\sphinxtableofcontents
\pagestyle{normal}
\phantomsection\label{\detokenize{index::doc}}



\chapter{Indices and tables}
\label{\detokenize{index:indices-and-tables}}\begin{itemize}
\item {} 
\DUrole{xref,std,std-ref}{genindex}

\item {} 
\DUrole{xref,std,std-ref}{modindex}

\item {} 
\DUrole{xref,std,std-ref}{search}

\end{itemize}


\chapter{Chapter on modules}
\label{\detokenize{index:module-tigre.algorithms}}\label{\detokenize{index:chapter-on-modules}}\index{tigre.algorithms (module)@\spxentry{tigre.algorithms}\spxextra{module}}\index{sirt() (in module tigre.algorithms)@\spxentry{sirt()}\spxextra{in module tigre.algorithms}}

\begin{fulllineitems}
\phantomsection\label{\detokenize{index:tigre.algorithms.sirt}}\pysiglinewithargsret{\sphinxcode{\sphinxupquote{tigre.algorithms.}}\sphinxbfcode{\sphinxupquote{sirt}}}{\emph{proj}, \emph{geo}, \emph{angles}, \emph{niter}, \emph{**kwargs}}{}
SART\_CBCT solves Cone Beam CT image reconstruction using Oriented Subsets
Simultaneous Algebraic Reconxtruction Techique algorithm
SIRT(PROJ,GEO,ALPHA,NITER) solves the reconstruction problem
using the projection data PROJ taken over ALPHA angles, corresponding
to the geometry descrived in GEO, using NITER iterations.
\begin{quote}
\begin{quote}\begin{description}
\item[{param proj}] \leavevmode
(np.ndarray, dtype=np.float32)

\end{description}\end{quote}

Input data, shape = (geo.nDector, nangles)
\begin{quote}\begin{description}
\item[{param geo}] \leavevmode
(tigre.geometry)

\end{description}\end{quote}

Geometry of detector and image (see examples/Demo code)
\begin{quote}\begin{description}
\item[{param angles}] \leavevmode
(np.ndarray , dtype=np.float32)

\end{description}\end{quote}

angles of projection, shape = (nangles,3)
\begin{quote}\begin{description}
\item[{param niter}] \leavevmode
(int)

\end{description}\end{quote}

number of iterations for reconstruction algorithm
\begin{quote}\begin{description}
\item[{param kwargs}] \leavevmode
(dict)

\end{description}\end{quote}

optional parameters
\begin{quote}\begin{description}
\item[{keyword blocksize}] \leavevmode
(int)
number of angles to be included in each iteration
of proj and backproj for OS\_SART

\item[{keyword lmbda}] \leavevmode
(np.float64)
Sets the value of the hyperparameter.

\item[{keyword lmbda\_red}] \leavevmode
(np.float64)
Reduction of lambda every iteration
lambda=lambdared*lambda. Default is 0.99

\item[{keyword init}] \leavevmode
(str)
Describes different initialization techniques.
\begin{quote}

“none”     : Initializes the image to zeros (default)
“FDK”      : intializes image to FDK reconstrucition
“multigrid”: Initializes image by solving the problem in
\begin{quote}

small scale and increasing it when relative
convergence is reached.
\end{quote}
\begin{description}
\item[{“image”}] \leavevmode{[}Initialization using a user specified{]}
image. Not recommended unless you really
know what you are doing.

\end{description}
\end{quote}

\item[{keyword InitImg}] \leavevmode
(np.ndarray)
Not yet implemented. Image for the “image” initialization.

\item[{keyword verbose}] \leavevmode
(Boolean)
Feedback print statements for algorithm progress
default=True

\item[{keyword Quameasopts}] \leavevmode
(list)
Asks the algorithm for a set of quality measurement
parameters. Input should contain a list or tuple of strings of
quality measurement names. Examples:
\begin{quote}

RMSE, CC, UQI, MSSIM
\end{quote}

\end{description}\end{quote}
\begin{description}
\item[{:keyword OrderStrategy}] \leavevmode{[}(str){]}\begin{description}
\item[{Chooses the subset ordering strategy. Options are:}] \leavevmode\begin{description}
\item[{“ordered”}] \leavevmode{[}uses them in the input order, but{]}
divided

\end{description}

“random”         : orders them randomply
“angularDistance”: chooses the next subset with the
\begin{quote}

biggest angular distance with the
ones used
\end{quote}

\end{description}

\end{description}

\begin{sphinxVerbatim}[commandchars=\\\{\}]
\PYG{g+gp}{\PYGZgt{}\PYGZgt{}\PYGZgt{} }\PYG{k+kn}{import} \PYG{n+nn}{numpy} \PYG{k}{as} \PYG{n+nn}{np}
\PYG{g+gp}{\PYGZgt{}\PYGZgt{}\PYGZgt{} }\PYG{k+kn}{import} \PYG{n+nn}{tigre}
\PYG{g+gp}{\PYGZgt{}\PYGZgt{}\PYGZgt{} }\PYG{k+kn}{import} \PYG{n+nn}{tigre}\PYG{n+nn}{.}\PYG{n+nn}{algorithms} \PYG{k}{as} \PYG{n+nn}{algs}
\PYG{g+gp}{\PYGZgt{}\PYGZgt{}\PYGZgt{} }\PYG{k+kn}{from} \PYG{n+nn}{tigre}\PYG{n+nn}{.}\PYG{n+nn}{demos}\PYG{n+nn}{.}\PYG{n+nn}{Test\PYGZus{}data} \PYG{k}{import} \PYG{n}{data\PYGZus{}loader}
\PYG{g+gp}{\PYGZgt{}\PYGZgt{}\PYGZgt{} }\PYG{n}{geo} \PYG{o}{=} \PYG{n}{tigre}\PYG{o}{.}\PYG{n}{geometry}\PYG{p}{(}\PYG{n}{mode}\PYG{o}{=}\PYG{l+s+s1}{\PYGZsq{}}\PYG{l+s+s1}{cone}\PYG{l+s+s1}{\PYGZsq{}}\PYG{p}{,}\PYG{n}{default\PYGZus{}geo}\PYG{o}{=}\PYG{k+kc}{True}\PYG{p}{,}
\PYG{g+gp}{\PYGZgt{}\PYGZgt{}\PYGZgt{} }                        \PYG{n}{nVoxel}\PYG{o}{=}\PYG{n}{np}\PYG{o}{.}\PYG{n}{array}\PYG{p}{(}\PYG{p}{[}\PYG{l+m+mi}{64}\PYG{p}{,}\PYG{l+m+mi}{64}\PYG{p}{,}\PYG{l+m+mi}{64}\PYG{p}{]}\PYG{p}{)}\PYG{p}{)}
\PYG{g+gp}{\PYGZgt{}\PYGZgt{}\PYGZgt{} }\PYG{n}{angles} \PYG{o}{=} \PYG{n}{np}\PYG{o}{.}\PYG{n}{linspace}\PYG{p}{(}\PYG{l+m+mi}{0}\PYG{p}{,}\PYG{l+m+mi}{2}\PYG{o}{*}\PYG{n}{np}\PYG{o}{.}\PYG{n}{pi}\PYG{p}{,}\PYG{l+m+mi}{100}\PYG{p}{)}
\PYG{g+gp}{\PYGZgt{}\PYGZgt{}\PYGZgt{} }\PYG{n}{src\PYGZus{}img} \PYG{o}{=} \PYG{n}{data\PYGZus{}loader}\PYG{o}{.}\PYG{n}{load\PYGZus{}head\PYGZus{}phantom}\PYG{p}{(}\PYG{n}{geo}\PYG{o}{.}\PYG{n}{nVoxel}\PYG{p}{)}
\PYG{g+gp}{\PYGZgt{}\PYGZgt{}\PYGZgt{} }\PYG{n}{proj} \PYG{o}{=} \PYG{n}{tigre}\PYG{o}{.}\PYG{n}{Ax}\PYG{p}{(}\PYG{n}{src\PYGZus{}img}\PYG{p}{,}\PYG{n}{geo}\PYG{p}{,}\PYG{n}{angles}\PYG{p}{)}
\PYG{g+gp}{\PYGZgt{}\PYGZgt{}\PYGZgt{} }\PYG{n}{output} \PYG{o}{=} \PYG{n}{algs}\PYG{o}{.}\PYG{n}{iterativereconalg}\PYG{p}{(}\PYG{n}{proj}\PYG{p}{,}\PYG{n}{geo}\PYG{p}{,}\PYG{n}{angles}\PYG{p}{,}\PYG{n}{niter}\PYG{o}{=}\PYG{l+m+mi}{50}
\PYG{g+gp}{\PYGZgt{}\PYGZgt{}\PYGZgt{} }                                \PYG{n}{blocksize}\PYG{o}{=}\PYG{l+m+mi}{20}\PYG{p}{)}
\end{sphinxVerbatim}

tigre.demos.run() to launch ipython notebook file with examples.
\end{quote}

\end{fulllineitems}

\index{sart() (in module tigre.algorithms)@\spxentry{sart()}\spxextra{in module tigre.algorithms}}

\begin{fulllineitems}
\phantomsection\label{\detokenize{index:tigre.algorithms.sart}}\pysiglinewithargsret{\sphinxcode{\sphinxupquote{tigre.algorithms.}}\sphinxbfcode{\sphinxupquote{sart}}}{\emph{proj}, \emph{geo}, \emph{angles}, \emph{niter}, \emph{**kwargs}}{}
SART\_CBCT solves Cone Beam CT image reconstruction using Oriented Subsets
Simultaneous Algebraic Reconstruction Techique algorithm
SART(PROJ,GEO,ALPHA,NITER) solves the reconstruction problem
using the projection data PROJ taken over ALPHA angles, corresponding
to the geometry described in GEO, using NITER iterations.
\begin{quote}
\begin{quote}\begin{description}
\item[{param proj}] \leavevmode
(np.ndarray, dtype=np.float32)

\end{description}\end{quote}

Input data, shape = (geo.nDector, nangles)
\begin{quote}\begin{description}
\item[{param geo}] \leavevmode
(tigre.geometry)

\end{description}\end{quote}

Geometry of detector and image (see examples/Demo code)
\begin{quote}\begin{description}
\item[{param angles}] \leavevmode
(np.ndarray , dtype=np.float32)

\end{description}\end{quote}

angles of projection, shape = (nangles,3)
\begin{quote}\begin{description}
\item[{param niter}] \leavevmode
(int)

\end{description}\end{quote}

number of iterations for reconstruction algorithm
\begin{quote}\begin{description}
\item[{param kwargs}] \leavevmode
(dict)

\end{description}\end{quote}

optional parameters
\begin{quote}\begin{description}
\item[{keyword blocksize}] \leavevmode
(int)
number of angles to be included in each iteration
of proj and backproj for OS\_SART

\item[{keyword lmbda}] \leavevmode
(np.float64)
Sets the value of the hyperparameter.

\item[{keyword lmbda\_red}] \leavevmode
(np.float64)
Reduction of lambda every iteration
lambda=lambdared*lambda. Default is 0.99

\item[{keyword init}] \leavevmode
(str)
Describes different initialization techniques.
\begin{quote}

“none”     : Initializes the image to zeros (default)
“FDK”      : intializes image to FDK reconstrucition
“multigrid”: Initializes image by solving the problem in
\begin{quote}

small scale and increasing it when relative
convergence is reached.
\end{quote}
\begin{description}
\item[{“image”}] \leavevmode{[}Initialization using a user specified{]}
image. Not recommended unless you really
know what you are doing.

\end{description}
\end{quote}

\item[{keyword InitImg}] \leavevmode
(np.ndarray)
Not yet implemented. Image for the “image” initialization.

\item[{keyword verbose}] \leavevmode
(Boolean)
Feedback print statements for algorithm progress
default=True

\item[{keyword Quameasopts}] \leavevmode
(list)
Asks the algorithm for a set of quality measurement
parameters. Input should contain a list or tuple of strings of
quality measurement names. Examples:
\begin{quote}

RMSE, CC, UQI, MSSIM
\end{quote}

\end{description}\end{quote}
\begin{description}
\item[{:keyword OrderStrategy}] \leavevmode{[}(str){]}\begin{description}
\item[{Chooses the subset ordering strategy. Options are:}] \leavevmode\begin{description}
\item[{“ordered”}] \leavevmode{[}uses them in the input order, but{]}
divided

\end{description}

“random”         : orders them randomply
“angularDistance”: chooses the next subset with the
\begin{quote}

biggest angular distance with the
ones used
\end{quote}

\end{description}

\end{description}

\begin{sphinxVerbatim}[commandchars=\\\{\}]
\PYG{g+gp}{\PYGZgt{}\PYGZgt{}\PYGZgt{} }\PYG{k+kn}{import} \PYG{n+nn}{numpy} \PYG{k}{as} \PYG{n+nn}{np}
\PYG{g+gp}{\PYGZgt{}\PYGZgt{}\PYGZgt{} }\PYG{k+kn}{import} \PYG{n+nn}{tigre}
\PYG{g+gp}{\PYGZgt{}\PYGZgt{}\PYGZgt{} }\PYG{k+kn}{import} \PYG{n+nn}{tigre}\PYG{n+nn}{.}\PYG{n+nn}{algorithms} \PYG{k}{as} \PYG{n+nn}{algs}
\PYG{g+gp}{\PYGZgt{}\PYGZgt{}\PYGZgt{} }\PYG{k+kn}{from} \PYG{n+nn}{tigre}\PYG{n+nn}{.}\PYG{n+nn}{demos}\PYG{n+nn}{.}\PYG{n+nn}{Test\PYGZus{}data} \PYG{k}{import} \PYG{n}{data\PYGZus{}loader}
\PYG{g+gp}{\PYGZgt{}\PYGZgt{}\PYGZgt{} }\PYG{n}{geo} \PYG{o}{=} \PYG{n}{tigre}\PYG{o}{.}\PYG{n}{geometry}\PYG{p}{(}\PYG{n}{mode}\PYG{o}{=}\PYG{l+s+s1}{\PYGZsq{}}\PYG{l+s+s1}{cone}\PYG{l+s+s1}{\PYGZsq{}}\PYG{p}{,}\PYG{n}{default\PYGZus{}geo}\PYG{o}{=}\PYG{k+kc}{True}\PYG{p}{,}
\PYG{g+gp}{\PYGZgt{}\PYGZgt{}\PYGZgt{} }                        \PYG{n}{nVoxel}\PYG{o}{=}\PYG{n}{np}\PYG{o}{.}\PYG{n}{array}\PYG{p}{(}\PYG{p}{[}\PYG{l+m+mi}{64}\PYG{p}{,}\PYG{l+m+mi}{64}\PYG{p}{,}\PYG{l+m+mi}{64}\PYG{p}{]}\PYG{p}{)}\PYG{p}{)}
\PYG{g+gp}{\PYGZgt{}\PYGZgt{}\PYGZgt{} }\PYG{n}{angles} \PYG{o}{=} \PYG{n}{np}\PYG{o}{.}\PYG{n}{linspace}\PYG{p}{(}\PYG{l+m+mi}{0}\PYG{p}{,}\PYG{l+m+mi}{2}\PYG{o}{*}\PYG{n}{np}\PYG{o}{.}\PYG{n}{pi}\PYG{p}{,}\PYG{l+m+mi}{100}\PYG{p}{)}
\PYG{g+gp}{\PYGZgt{}\PYGZgt{}\PYGZgt{} }\PYG{n}{src\PYGZus{}img} \PYG{o}{=} \PYG{n}{data\PYGZus{}loader}\PYG{o}{.}\PYG{n}{load\PYGZus{}head\PYGZus{}phantom}\PYG{p}{(}\PYG{n}{geo}\PYG{o}{.}\PYG{n}{nVoxel}\PYG{p}{)}
\PYG{g+gp}{\PYGZgt{}\PYGZgt{}\PYGZgt{} }\PYG{n}{proj} \PYG{o}{=} \PYG{n}{tigre}\PYG{o}{.}\PYG{n}{Ax}\PYG{p}{(}\PYG{n}{src\PYGZus{}img}\PYG{p}{,}\PYG{n}{geo}\PYG{p}{,}\PYG{n}{angles}\PYG{p}{)}
\PYG{g+gp}{\PYGZgt{}\PYGZgt{}\PYGZgt{} }\PYG{n}{output} \PYG{o}{=} \PYG{n}{algs}\PYG{o}{.}\PYG{n}{iterativereconalg}\PYG{p}{(}\PYG{n}{proj}\PYG{p}{,}\PYG{n}{geo}\PYG{p}{,}\PYG{n}{angles}\PYG{p}{,}\PYG{n}{niter}\PYG{o}{=}\PYG{l+m+mi}{50}
\PYG{g+gp}{\PYGZgt{}\PYGZgt{}\PYGZgt{} }                                \PYG{n}{blocksize}\PYG{o}{=}\PYG{l+m+mi}{20}\PYG{p}{)}
\end{sphinxVerbatim}

tigre.demos.run() to launch ipython notebook file with examples.
\end{quote}

\end{fulllineitems}



\renewcommand{\indexname}{Python Module Index}
\begin{sphinxtheindex}
\let\bigletter\sphinxstyleindexlettergroup
\bigletter{t}
\item\relax\sphinxstyleindexentry{tigre.algorithms}\sphinxstyleindexpageref{index:\detokenize{module-tigre.algorithms}}
\end{sphinxtheindex}

\renewcommand{\indexname}{Index}
\printindex
\end{document}